\documentclass[11pt,titlepage]{article}
\usepackage{ucs}
\usepackage[utf8x]{inputenc}
\usepackage[T1]{fontenc}
\usepackage[ngerman]{babel}
\usepackage{graphicx}
\usepackage{titlesec}
\usepackage{url}
\usepackage{lastpage}
\usepackage{listings}
\usepackage{color}
\usepackage{fancyhdr}
\usepackage{geometry}
\usepackage{wrapfig}
\usepackage{float}
\usepackage{subcaption}
\usepackage{hyperref}
\usepackage{ragged2e}
\usepackage{framed}
\usepackage{quoting}
% remove current style and use fancyplain
\pagestyle{fancyplain}
\fancyhf{}
% remove rule/lines as well
\renewcommand{\headrulewidth}{0pt} 
\renewcommand{\footrulewidth}{0pt}
% set papersize, magin and footersize
\geometry{a4paper,portrait,left={3cm},right={3cm},top={2cm},bottom={1cm},includefoot,foot={1cm}}
% set footer
\rfoot{Seite \thepage \hspace{1pt} von \pageref{LastPage}}
% define some colors
\definecolor{lightgray}{rgb}{.95,.95,.95}
\definecolor{shadecolor}{rgb}{.95,.95,.95}
\definecolor{darkgray}{rgb}{.4,.4,.4}
\definecolor{purple}{rgb}{0.65, 0.12, 0.82}
% set color and font of ''\url''
\renewcommand\UrlFont{\color{blue}\rmfamily\itshape}
% colorbox which can wrap lines
\newcommand\code[1]{\codehelp#1 \relax\relax}
\def\codehelp#1 #2\relax{\allowbreak\grayspace\codecolor{#1}\ifx\relax#2\else
 \codehelp#2\relax\fi}
\newcommand\codecolor[1]{\colorbox{lightgray}{\textcolor{black}{%
  \ttfamily\mystrut\smash{\detokenize{#1}}}}}
\def\mystrut{\rule[\dimexpr-\dp\strutbox+\fboxsep]{0pt}{%
 \dimexpr\normalbaselineskip-2\fboxsep}}
\def\grayspace{\hspace{0pt minus \fboxsep}}
% add ''\code'' to highligth single code lines
%\newcommand{\code}[1]{\wrapcolorbox[lightgray]{\ttfamily{#1}}}

% add ''\shadedquotation'' to highligth quoates
\newenvironment{shadedquotation}
 {\begin{shaded*}
  \quoting[leftmargin=0pt, vskip=0pt]
 }
 {\endquoting
 \end{shaded*}
}

% define ''JavaScript'' as a language for enviroment ''lstlisting''
\lstdefinelanguage{JavaScript}{
  keywords={typeof, new, true, false, catch, function, return, null, catch, switch, var, if, in, while, do, else, case, break},
  keywordstyle=\color{blue}\bfseries,
  ndkeywords={class, export, boolean, throw, implements, import, this},
  ndkeywordstyle=\color{darkgray}\bfseries,
  identifierstyle=\color{black},
  sensitive=false,
  comment=[l]{//},
  morecomment=[s]{/*}{*/},
  commentstyle=\color{purple}\ttfamily,
  stringstyle=\color{red}\ttfamily,
  morestring=[b]',
  morestring=[b]''
}

\lstset{
   language=JavaScript,
   backgroundcolor=\color{lightgray},
   extendedchars=true,
   basicstyle=\footnotesize\ttfamily,
   showstringspaces=false,
   showspaces=false,
   numbers=left,
   numberstyle=\footnotesize,
   numbersep=9pt,
   tabsize=2,
   breaklines=true,
   showtabs=false,
   captionpos=b
}
% set title
\title{ANFORDERUNGS-ANALYSE ZUR CLOUD}
\author{Markus Gachnang und Martin Sprecher}
\date{\today{}}
% set parindent to 0px to remove it (Einrücken von neuer Absatz)
\setlength\parindent{0pt}
% ---------------------------------------------------------------------------
% begin Document 
\begin{document}
% set font
\sffamily
% print title
\maketitle
\newpage
% print index
\tableofcontents{}
\setcounter{page}{1}
\newpage
% linksbündig
\RaggedRight
% kein brechen von Wörtern
\tolerance=1
\emergencystretch=\maxdimen
\hyphenpenalty=10000
\hbadness=10000

\section{Aufgabe 1}
\label{sec:Aufgabe-1}

\begin{shadedquotation}
  Welches sind die Argumente pro / contra Cloud im Allgemeinen aus Sicht des Endkunden sowie aus Sicht des Providers.
\end{shadedquotation}

\par\medskip

\begin{tabular}{ |p{7cm}|p{7cm}|  }
  \hline
  \multicolumn{2}{|c|}{Sicht des Endkunden} \\
  \hline
  Pro & Contra \\
  \hline
  \begin{itemize}
    \item First Pro
    \item Second Pro
  \end{itemize}
  & 
  \begin{itemize}
    \item First Contra
    \item Second Contra
  \end{itemize}
  \\
  \hline
\end{tabular}

\par\medskip

\begin{tabular}{ |p{7cm}|p{7cm}|  }
  \hline
  \multicolumn{2}{|c|}{Sicht des Providers} \\
  \hline
  Pro & Contra \\
  \hline
  \begin{itemize}
    \item First Pro
    \item Second Pro
  \end{itemize}
  & 
  \begin{itemize}
    \item First Contra
    \item Second Contra
  \end{itemize}
  \\
  \hline
\end{tabular}

\par\medskip

\section{Aufgabe 2}
\label{sec:Aufgabe-2}

\begin{shadedquotation}
  Vergleichen Sie die beiden Varianten ''Public'' vs. ''Private'' Cloud.
\end{shadedquotation}

\par\medskip

\section{Aufgabe 3}
\label{sec:Aufgabe-3}

\begin{shadedquotation}
  Nennen Sie Beispiele oder Use-cases, die sich besonders für die Public oder Private Cloud eignen.
\end{shadedquotation}

\subsection{Applikationsebene}
\label{subsec:Aufgabe-3_Applikationsebene}

\subsection{Plattformebene}
\label{subsec:Aufgabe-3_Plattformebene}

\subsection{Infrastrukturebene}
\label{subsec:Aufgabe-3_Infrastrukturebene}

\section{Aufgabe 4}
\label{sec:Aufgabe-4}
\begin{shadedquotation}
  Sie möchten eine Startup gründen, welche Cloud-Services anbietet. Um gegenüber weltweiten
  Anbietern einen Konkurrenzvorteil zu haben, wollen Sie sämtliche Leistungen in der Schweiz
  produzieren und auch alle Daten sicher in einem Bunker in den Bergen lagern. Sie beginnen mit
  einfacheren Infrastruktur Services (Compute, Storage), welches Sie an kleine und grosse Firmen
  anbieten wollen. Nun überlegen sie sich, was sich gegenüber einer «klassischen» Inhouse IT alles ändert,
  wenn man daraus Cloud-Dienste baut. Beschreiben Sie die technischen Anforderungen an eine Cloud
  Infrastruktur. Nehmen Sie die Definitionen von Cloud wie z.B. OSSM und überlegen Sie sich, welche
  technischen Anforderungen sich aus diesen ergeben:


  \begin{itemize}
    \item Welche Anforderungen müssen Applikationen erfüllen, damit sie in die Cloud «verschoben»
    werden können?
    \item Welche neuen zusätzlichen technischen Anforderungen ergeben sich, wenn aus traditioneller IT
    ein Cloud fähiges Rechenzentrum entwickelt werden soll. Was muss die Netzwerk-, Compute
    und Storage Infrastruktur erfüllen, damit Cloud Dienste an eine Vielzahl von Kunden angeboten
    werden können?
  \end{itemize}

  Verwenden Sie die folgende Tabelle. Als Spalten verwenden Sie wichtige Cloud Eigenschaften (die für
  alle Bereiche gelten). Erweitern Sie die Tabelle um solche allgemein gültigen Eigenschaften. Dann
  beschreiben Sie die Auswirkungen dieser Eigenschaften auf Applikationen, Netz, Compute und Storage.
\end{shadedquotation}

\begin{tabular}{ |p{2cm}|p{3cm}|p{3cm}|p{3cm}|p{3cm}|  }
  \hline
  & Mandantenfähigkeit & Mobility & … & . \\
  \hline
  Applications & & & & \\
  Netz & & & & \\
  Compute & & & & \\
  Storage & & & & \\
\end{tabular}

\end{document}