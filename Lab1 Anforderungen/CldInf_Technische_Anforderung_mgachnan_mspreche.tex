\documentclass[11pt,titlepage]{article}
\usepackage{ucs}
\usepackage[utf8x]{inputenc}
\usepackage[T1]{fontenc}
\usepackage[ngerman]{babel}
\usepackage{graphicx}
\usepackage{titlesec}
\usepackage{url}
\usepackage{lastpage}
\usepackage{listings}
\usepackage{color}
\usepackage{fancyhdr}
\usepackage{geometry}
\usepackage{wrapfig}
\usepackage{float}
\usepackage{subcaption}
\usepackage{hyperref}
\usepackage{ragged2e}
\usepackage{framed}
\usepackage{quoting}
% remove current style and use fancyplain
\pagestyle{fancyplain}
\fancyhf{}
% remove rule/lines as well
\renewcommand{\headrulewidth}{0pt}
\renewcommand{\footrulewidth}{0pt}
% set papersize, magin and footersize
\geometry{a4paper,portrait,left={3cm},right={3cm},top={2cm},bottom={1cm},includefoot,foot={1cm}}
% set footer
\rfoot{Seite \thepage \hspace{1pt} von \pageref{LastPage}}
% define some colors
\definecolor{lightgray}{rgb}{.95,.95,.95}
\definecolor{shadecolor}{rgb}{.95,.95,.95}
\definecolor{darkgray}{rgb}{.4,.4,.4}
\definecolor{purple}{rgb}{0.65, 0.12, 0.82}
% set color and font of ''\url''
\renewcommand\UrlFont{\color{blue}\rmfamily\itshape}
% colorbox which can wrap lines
\newcommand\code[1]{\codehelp#1 \relax\relax}
\def\codehelp#1 #2\relax{\allowbreak\grayspace\codecolor{#1}\ifx\relax#2\else
 \codehelp#2\relax\fi}
\newcommand\codecolor[1]{\colorbox{lightgray}{\textcolor{black}{%
  \ttfamily\mystrut\smash{\detokenize{#1}}}}}
\def\mystrut{\rule[\dimexpr-\dp\strutbox+\fboxsep]{0pt}{%
 \dimexpr\normalbaselineskip-2\fboxsep}}
\def\grayspace{\hspace{0pt minus \fboxsep}}
% add ''\code'' to highligth single code lines
%\newcommand{\code}[1]{\wrapcolorbox[lightgray]{\ttfamily{#1}}}

% add ''\shadedquotation'' to highligth quoates
\newenvironment{shadedquotation}
 {\begin{shaded*}
  \quoting[leftmargin=0pt, vskip=0pt]
 }
 {\endquoting
 \end{shaded*}
}

% define ''JavaScript'' as a language for enviroment ''lstlisting''
\lstdefinelanguage{JavaScript}{
  keywords={typeof, new, true, false, catch, function, return, null, catch, switch, var, if, in, while, do, else, case, break},
  keywordstyle=\color{blue}\bfseries,
  ndkeywords={class, export, boolean, throw, implements, import, this},
  ndkeywordstyle=\color{darkgray}\bfseries,
  identifierstyle=\color{black},
  sensitive=false,
  comment=[l]{//},
  morecomment=[s]{/*}{*/},
  commentstyle=\color{purple}\ttfamily,
  stringstyle=\color{red}\ttfamily,
  morestring=[b]',
  morestring=[b]''
}

\lstset{
   language=JavaScript,
   backgroundcolor=\color{lightgray},
   extendedchars=true,
   basicstyle=\footnotesize\ttfamily,
   showstringspaces=false,
   showspaces=false,
   numbers=left,
   numberstyle=\footnotesize,
   numbersep=9pt,
   tabsize=2,
   breaklines=true,
   showtabs=false,
   captionpos=b
}
% set title
\title{ANFORDERUNGS-ANALYSE ZUR CLOUD}
\author{Markus Gachnang und Martin Sprecher}
\date{\today{}}
% set parindent to 0px to remove it (Einrücken von neuer Absatz)
\setlength\parindent{0pt}
% ---------------------------------------------------------------------------
% begin Document
\begin{document}
% set font
\sffamily
% print title
\maketitle
\newpage
% print index
\tableofcontents{}
\setcounter{page}{1}
\newpage
% linksbündig
\RaggedRight
% kein brechen von Wörtern
\tolerance=1
\emergencystretch=\maxdimen
\hyphenpenalty=10000
\hbadness=10000
\section{Cloud nutzen Analyse}
\subsection{Aufgabe 1}
\label{sec:Aufgabe-1}

\begin{shadedquotation}
  Welches sind die Argumente pro / contra Cloud im Allgemeinen aus Sicht des Endkunden sowie aus Sicht des Providers.
\end{shadedquotation}

\par\medskip

\begin{tabular}{ |p{7cm}|p{7cm}|  }
  \hline
  \multicolumn{2}{|c|}{Sicht des Endkunden} \\
  \hline
  Pro & Contra \\
  \hline
  \begin{itemize}
    \item Kostenreduktion
    \item Einsparung von IT-Administrationsaufwand
    \item Garantierte Verfügbarkeit
    \item Schnelles skalieren / Agilität (on-demand)
  \end{itemize}
  &
  \begin{itemize}
    \item Daten-Kontrolle / -Sicherheit
    \item Abhänigkeit zum Provider
    \item Latenz / langsamer Zugriff übers Internet
    \item Ohne Internet kein Dienst
  \end{itemize}
  \\
  \hline
\end{tabular}

\par\medskip

Für den Endkunden ergibt eine externe Cloud eine Kostenreduktion durch die nicht benötigte Hardware, dessen Betrieb und einen reduzierten IT-Administrationsaufwand.

Normalerweise garantiert ein Provider dem Kunden die Verfügbarkeit und sorgt bei Unterbruch automatisch und selbständig für die Bereinigung.

Zusätzlich bietet dieser eine Platform an, auf welcher die Cloud ''on-demand'' angepasst (skaliert) werden kann.

\par\medskip

Im Gegenzug befinden sich die Daten des Endkunden auf der Cloud und muss sich auf den Provider verlassen, dass diese sicher (vor Angreifern / Backup) sind. Man könnte es auch als Vorteil sehen, selber keine Sicherheit für die Daten bieten zu müssen, falls der Provider eine gute Sicherheit bietet.

Ausserdem ist die Cloud nicht im eigenem Netzwerk sondern muss übers Internet angesprochen werden. Dadurch können Latenzen enstehen und ohne Internet hat man keinen Zugriff mehr.

\par\medskip\medskip

\begin{tabular}{ |p{7cm}|p{7cm}|  }
  \hline
  \multicolumn{2}{|c|}{Sicht des Providers} \\
  \hline
  Pro & Contra \\
  \hline
  \begin{itemize}
    \item Abo-Modell / Money
    \item Automatisches System
  \end{itemize}
  &
  \begin{itemize}
    \item Mehr Leistung verfügbar als benötigt
    \item IT-Administrationsaufwand / Support / Controlling
    \item Internet Bandbreite
    \item Verfügbarkeit
    \item Datensicherheit
  \end{itemize}
  \\
  \hline
\end{tabular}

\par\medskip

Für den Provider ist das Anbieten von Clouds eine Einnahmequelle, welche meist im Abo-Modell (FixKosten / per CPU-Aufwand) angeboten werden.
Durch ein gutes System muss der Provider kaum Hand an die Infrastruktur legen, da das meiste Automatisch managed wird.

\par\medskip

Wegen des Angebots der Skalierung muss immer mehr Hardware / Leistung zur Verfügung stehen als schlussendlich gebraucht wird und obwohl das System vieles automatisch erledigen kann, sind IT-Fachleute nötig, um neue Hardware anzuschliesen, das System zu überwachen oder dem Kunden Support zu liefern.
Die Cloud-Infrastruktur muss eine genügend gute Bandbreite liefern, um alle Clouds eine schnelle und reibungslose Kommunikation zu ermöglichen.
Auch ein Backup oder Ausfallsicherung sollte dem Kunden geboten werden, um die versprochene Verfügbarkeit und Datensicherheit zu gewährleisten.

\par\medskip

\subsection{Aufgabe 2}
\label{sec:Aufgabe-2}

\begin{shadedquotation}
  Vergleichen Sie die beiden Varianten ''Public'' vs. ''Private'' Cloud.
\end{shadedquotation}

\par\medskip

\begin{tabular}{ |p{7cm}|p{7cm}|  }
  \hline
  Public & Private \\
  \hline
  Die Cloud wird bei einem Provider gehostet &
  Die Cloud wird ''in house'' bei sich selber gehostet \par \\

  Die Hardware und den Betreib wird vom Provider geleistet &
  Die Hardware muss von einem selbst beschafft und betrieben werden \par \\

  Skalierung ist problemlos möglich &
  Um höher zu skalieren, muss eventuell neue Hardware beschafft und eingerichtet werden \par \\

  Verfügbarkeit und genügend Bandbreite wird vom Provider garantiert (Controlling) &
  Man muss selber überprüfen, ob die Cloud läuft und genügend Bandbreite hat \par \\

  \hline
\end{tabular}

  \par\medskip

\subsection{Aufgabe 3}
\label{sec:Aufgabe-3}

\begin{shadedquotation}
  Nennen Sie Beispiele oder Use-cases, die sich besonders für die Public oder Private Cloud eignen.
\end{shadedquotation}

\subsubsection{Applikationsebene}
\label{subsec:Aufgabe-3_Applikationsebene}
{\large\bf Software as a Service (SaaS)} \\\medskip

Eine einzelne Homepage (oder WebApplikation) wird typischerweise auf einer ''Public'' Cloud gehostet, um die Verfügbarkeit zu garantieren.
\medskip

Ein BackupSystem (z.B. NAS) sollte ''Privat'' gehostet werden, da grössere Datenmengen effizienter über ein lokales Netzwerk übertragen werden können als über das Internet.

\subsubsection{Plattformebene}
\label{subsec:Aufgabe-3_Plattformebene}
{\large\bf Platform as a Service (PaaS)} \\\medskip
\colorbox{yellow}{TODO: EXAMPLE FOR ''Public''.}
\par\medskip

Mehrere Homepages (oder WebApplikationen) oder Datenbanken benötigen eine Plattform (Hosting-Dienst wie IIS oder Apache, Datenbank-Server wie MSSQL, MYSQL oder Oracle).
Im ''Private'' wird typischerweise ein IaaS gehostet um nicht mehrere Instanzen des Services laufen zu lassen, dadurch werden Ressourcen gespart.

''Public'' bieten normalerweise einem spezifische SaaS an, um eine Homepage oder Datenbank zu hosten, damit sie vom gleichem Prinzip profitieren können.

\subsubsection{Infrastrukturebene}
\label{subsec:Aufgabe-3_Infrastrukturebene}
{\large\bf Infrastruktur as a Service (IaaS)} \\\medskip
\colorbox{yellow}{TODO: EXAMPLE FOR ''Public''.}
Die Bewältigung komplexer Aufgaben, mit mehreren Millionen Variablen oder Berechnungen, erfordert normalerweise die Verwendung von Supercomputern oder Clustern. Hier kann Public IaaS aufgrund seiner Skalierbarkeit eine bessere Alternative sein.
Einer der Bausteine des modernen Marketings ist das Sammeln großer Mengen von Benutzerdaten. Die Verarbeitung ist jedoch wichtiger als nur das Sammeln dieser Informationen. IaaS kann Big Data verwalten, speichern und analysieren. Hier kann Public IaaS aufgrund seiner Skalierbarkeit eine bessere Alternative sein.
\par\medskip
\colorbox{yellow}{TODO: EXAMPLE FOR ''Privat''.}

\section{Technische Anforderungen an die Cloud}
\subsection{Aufgabe 4}
\label{sec:Aufgabe-4}
\begin{shadedquotation}
  Sie möchten eine Startup gründen, welche Cloud-Services anbietet. Um gegenüber weltweiten
  Anbietern einen Konkurrenzvorteil zu haben, wollen Sie sämtliche Leistungen in der Schweiz
  produzieren und auch alle Daten sicher in einem Bunker in den Bergen lagern. Sie beginnen mit
  einfacheren Infrastruktur Services (Compute, Storage), welches Sie an kleine und grosse Firmen
  anbieten wollen. Nun überlegen sie sich, was sich gegenüber einer «klassischen» Inhouse IT alles ändert,
  wenn man daraus Cloud-Dienste baut. Beschreiben Sie die technischen Anforderungen an eine Cloud
  Infrastruktur. Nehmen Sie die Definitionen von Cloud wie z.B. OSSM und überlegen Sie sich, welche
  technischen Anforderungen sich aus diesen ergeben:

  \begin{itemize}
    \item Welche Anforderungen müssen Applikationen erfüllen, damit sie in die Cloud «verschoben» werden können?
    \item Welche neuen zusätzlichen technischen Anforderungen ergeben sich, wenn aus traditioneller IT
    ein Cloud fähiges Rechenzentrum entwickelt werden soll. Was muss die Netzwerk-, Compute
    und Storage Infrastruktur erfüllen, damit Cloud Dienste an eine Vielzahl von Kunden angeboten
    werden können?
  \end{itemize}

  Verwenden Sie die folgende Tabelle. Als Spalten verwenden Sie wichtige Cloud Eigenschaften (die für
  alle Bereiche gelten). Erweitern Sie die Tabelle um solche allgemein gültigen Eigenschaften. Dann
  beschreiben Sie die Auswirkungen dieser Eigenschaften auf Applikationen, Netz, Compute und Storage.
\end{shadedquotation}

\subsubsection{Technischen Anforderungen}
\label{sec:Aufgabe-4_technic}
\begin{shadedquotation}
  Welche Anforderungen müssen Applikationen erfüllen, damit sie in die Cloud «verschoben» werden können?
\end{shadedquotation}

Die Applikation muss über ein Netzwerk angesprochen werden können. Desktop-Applikationen wie ''Word'' können nicht auf einer Cloud betrieben werden.
Microsoft musste ''Word'' neu als WebApplikation realisieren, um es in der Cloud laufen zu lassen.
\par\medskip
Applikationen, die eine spezifische Hardware benötigen (welche direkt angesteuert werden müssen) sind ebenfalls nicht Cloud fähig. Die Hardware müsste in ein IoT-Device umgebaut werden, so dass ein ansteuern übers Netzwerk möglich wird.
Zum Beispiel gibt es als Kopierschutz CDs oder Dongles, welches direkt an der Cloud eingelegt / angeschlossen werden muss. Eine ''Public'' Cloud erlaubt einem normalerweise nicht, ein solches Gerät anzuschliessen.
Mit einer ''Private'' Cloud könnte dies möglich sein, wenn die Virtualisierung dies erlaubt, aber einen grösseren Aufwand bedeuten (Eine Ausfallsteuerung wie Cloud auf andere Hardware verschieben wird dadurch unmöglich!).

\label{sec:Aufgabe-4_technic}
\begin{shadedquotation}
  Welche neuen zusätzlichen technischen Anforderungen ergeben sich, wenn aus traditioneller IT
    ein Cloud fähiges Rechenzentrum entwickelt werden soll. Was muss die Netzwerk-, Compute
    und Storage Infrastruktur erfüllen, damit Cloud Dienste an eine Vielzahl von Kunden angeboten
    werden können?
\end{shadedquotation}

Zuerst ist die Hardware erforderlich. Eine einzige Maschine ist wegen des Ausfallschutzes nicht ausreichend. Ein USV (Unterbrechungsfreie Stromversorgung) ist zu empfehlen.
\par\medskip
Die Maschinen müssen miteinander im Netzwerk verbunden sein und eine Virtualisierungsumgebung muss auf jeder eingerichtet werden.
\par\medskip
Die Virtualisierungsumgebung sollte die OSSM (On-Demand, Self-Service, Scalable und Measurable) erfüllen, es gibt es verschiedene Optionen, auch open-source, einige Beispiele wären etwa:
\begin{itemize}
  \item Openstack (IaaS)
  \item Proxmox (PaaS)
  \item Cloudfoundry (PaaS, SaaS)
  \item Apache CloudStack (IaaS)
\end{itemize}
Die meisten Virtualisierungsumgebung bieten selber kein Zahl- und Abrechnungs-System an. Dies muss eventuell selber realisiert werden oder eine passende Virtualisierungsumgebung gefunden werden.
\par\medskip
Ein Backup-System für das Sichern der Daten der Kunden sollte ebenfalls realisiert werden.
\par\medskip
Eine fixe IP und Domäne wird auch nötig sein, damit der Kunde Zugriff auf den Service erhält.
Neben Internet, Strom und Kühlung sollte dies alles sein, um ein Cloud-Service anbieten zu können.

\subsection{Tabelle}
\label{sec:Aufgabe-4_tabelle}
\colorbox{yellow}{TODO}
\begin{tabular}{ |p{2cm}|p{3cm}|p{3cm}|p{3cm}|p{3cm}|  }
  \hline
  & Mandantenfähigkeit & Mobility & … & . \\
  \hline
  Applications & & & & \\
  Netz & & & & \\
  Compute & & & & \\
  Storage & & & & \\
  \hline
\end{tabular}

\end{document}
