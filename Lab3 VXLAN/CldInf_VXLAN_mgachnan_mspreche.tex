\documentclass[11pt,titlepage]{article}
\usepackage{ucs}
\usepackage[utf8x]{inputenc}
\usepackage[T1]{fontenc}
\usepackage[ngerman]{babel}
\usepackage{graphicx}
\usepackage{titlesec}
\usepackage{url}
\usepackage{lastpage}
\usepackage{listings}
\usepackage{color}
\usepackage{fancyhdr}
\usepackage{geometry}
\usepackage{wrapfig}
\usepackage{float}
\usepackage{subcaption}
\usepackage{hyperref}
\hypersetup{
    colorlinks=true,
    linkcolor=black,
    anchorcolor=black,
	citecolor=black,
	filecolor=black,
	menucolor=black,
	runcolor=black,
    urlcolor=blue,
	linktoc=all,
    pdftitle={Linux Networking},
    pdfauthor={Markus Gachnang und Martin Sprecher}
}
\usepackage{ragged2e}
\usepackage{framed}
\usepackage{quoting}
\usepackage{lscape}
\usepackage[table]{xcolor}
\usepackage{graphicx} 
\usepackage{pdfpages}
% remove current style and use fancyplain
\pagestyle{fancyplain}
\fancyhf{}
% remove rule/lines as well
\renewcommand{\headrulewidth}{0pt}
\renewcommand{\footrulewidth}{0pt}
% set papersize, magin and footersize
\geometry{a4paper,portrait,left={3cm},right={3cm},top={2cm},bottom={1cm},includefoot,foot={1cm}}
% set footer
\rfoot{Seite \thepage \hspace{1pt} von \pageref{LastPage}}
% Bibliographie
\usepackage{cite}
\def\BibTeX{{\rm B\kern-.05em{\sc i\kern-.025em b}\kern-.08em
    T\kern-.1667em\lower.7ex\hbox{E}\kern-.125emX}}
% define some colors
\definecolor{lightgray}{rgb}{.95,.95,.95}
\definecolor{shadecolor}{rgb}{.95,.95,.95}
\definecolor{darkgray}{rgb}{.4,.4,.4}
\definecolor{purple}{rgb}{0.65, 0.12, 0.82}
% set color and font of ''\url''
\renewcommand\UrlFont{\color{blue}\rmfamily\itshape}
% colorbox which can wrap lines
\newcommand\code[1]{\codehelp#1 \relax\relax}
\def\codehelp#1 #2\relax{\allowbreak\grayspace\codecolor{#1}\ifx\relax#2\else
 \codehelp#2\relax\fi}
\newcommand\codecolor[1]{\colorbox{lightgray}{\textcolor{black}{%
  \ttfamily\mystrut\smash{\detokenize{#1}}}}}
\def\mystrut{\rule[\dimexpr-\dp\strutbox+\fboxsep]{0pt}{%
 \dimexpr\normalbaselineskip-2\fboxsep}}
\def\grayspace{\hspace{0pt minus \fboxsep}}
% add ''\code'' to highligth single code lines
%\newcommand{\code}[1]{\wrapcolorbox[lightgray]{\ttfamily{#1}}}

% add ''\shadedquotation'' to highligth quoates
\newenvironment{shadedquotation}
 {\begin{shaded*}
  \quoting[leftmargin=0pt, vskip=0pt]
 }
 {\endquoting
 \end{shaded*}
}

% define ''JavaScript'' as a language for enviroment ''lstlisting''
\lstdefinelanguage{JavaScript}{
  keywords={typeof, new, true, false, catch, function, return, null, catch, switch, var, if, in, while, do, else, case, break},
  keywordstyle=\color{blue}\bfseries,
  ndkeywords={class, export, boolean, throw, implements, import, this},
  ndkeywordstyle=\color{darkgray}\bfseries,
  identifierstyle=\color{black},
  sensitive=false,
  comment=[l]{//},
  morecomment=[s]{/*}{*/},
  commentstyle=\color{purple}\ttfamily,
  stringstyle=\color{red}\ttfamily,
  morestring=[b]',
  morestring=[b]''
}

\lstset{
   language=JavaScript,
   backgroundcolor=\color{lightgray},
   extendedchars=true,
   basicstyle=\footnotesize\ttfamily,
   showstringspaces=false,
   showspaces=false,
   numbers=left,
   numberstyle=\footnotesize,
   numbersep=9pt,
   tabsize=2,
   breaklines=true,
   showtabs=false,
   captionpos=b
}
% set title
\title{VXLAN und eVPN}
\author{Markus Gachnang und Martin Sprecher}
\date{\today{}}
% set parindent to 0px to remove it (Einrücken von neuer Absatz)
\setlength\parindent{0pt}
% ---------------------------------------------------------------------------
% begin Document
\begin{document}
% set font
\sffamily
% print title
\maketitle
\newpage
% print index
\tableofcontents{}
\setcounter{page}{1}
\newpage
% linksbündig
\RaggedRight
% kein brechen von Wörtern
\tolerance=1
\emergencystretch=\maxdimen
\hyphenpenalty=10000
\hbadness=10000
% medskip before section
\let\SectionOriginal\section
\renewcommand\section[1]{\par\medskip\SectionOriginal{#1}}
% medskip before subsection
\let\SubSectionOriginal\subsection
\renewcommand\subsection[1]{\par\medskip\SubSectionOriginal{#1}}

\section{Config}
\label{sec:Config}
Prüfe die OSPF Einstellung bei Spine-1
\begin{shadedquotation}
	show ip route ospf \\
	ping 10.0.128.5
\end{shadedquotation}
Prüfe die OSPF Einstellung bei Spine-2
\begin{shadedquotation}
	show ip route ospf \\
	ping 10.0.0.21
\end{shadedquotation}
\subsubsection{Flood and Learn}
\begin{enumerate}
	\item Configure Multicast: The first step is it to configure PIM-ASM with a rendez-vous point on the spine switches.
	\begin{itemize}
		\item use the spines loopback 1 as rendez-vous point
		\item configure Spine and Leaf interfaces for PIM
	\end{itemize}
	Spine-1
	\begin{shadedquotation}
		configure terminal \\
		feature pim \\
		interface loopback1 \\
		ip pim sparse-mode \\
		ip pim RP-address 192.168.0.100 \\
		interface ethernet1/1-4 \\
		ip pim sparse-mode \\
	\end{shadedquotation}
	Spine-2
	\begin{shadedquotation}
		feature pim \\
		interface loopback1 \\
		ip pim sparse-mode \\
		ip pim RP-address 192.168.0.100 \\
		interface ethernet1/1-4 \\
		ip pim sparse-mode
	\end{shadedquotation}
	Leaf1-4
	\begin{shadedquotation}
		feature pim \\
		interface Ethernet1/1-2 \\
		ip pim sparse-mode \\
		ip pim RP-address 192.168.0.100 \\
		interface loopback1 \\
		ip pim sparse-mode
	\end{shadedquotation}
	Überprüfe Multicast
	\begin{shadedquotation}
		show ip mroute \\
		show ip pim neighbor \\
	\end{shadedquotation}
\item The next step will be the creation of the VLANs and the virtual network identifiers. The VNIs need to be linked to the VLANs. Use the VLAN-ID 140 and as VNI 50140.
	\begin{shadedquotation}
		feature vn-segment-vlan-based \\
		vlan 140 \\
		vn-segment 50140
	\end{shadedquotation}
\item Now you have to create the VXLAN tunnel interfaces. Don’t forget to add the tunnel endpoints to the VNI and multicast group. All VTEPs in VLAN 140 should belong to the multicast-group 239.0.0.140. \\
	Erstelle VXLAN auf allen Leafes
	\begin{shadedquotation}
		feature nv overlay \\
		interface nve1 \\
		source-interface loopback1 \\
		member vni 50140 mcast-group 239.0.0.140 \\
		no shutdown
	\end{shadedquotation}
	Kontrolle
	\begin{shadedquotation}
		show nve peers detail \\
		show nve vni 50140
	\end{shadedquotation}
\end{enumerate}

\end{document}
